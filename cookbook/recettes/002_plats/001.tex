\begin{nouvellerecette}
\recette{Cannelés bordelais}
\temps{15 minutes + 24 heures}{1 heure}

\ingredients[(pour 40 mini cannelés)]{
\item 1/2 litre de lait
\item 1 pincée de sel
\item 2 œufs entiers et 2 jaunes
\item 1/2 gousse de vanille
\item 4 cuillères à soupe de rhum
\item 100g de farine
\item 250 g de sucre en poudre
\item 50g de beurre
}

Faire bouillir le lait avec la vanille et le beurre.
Pendant ce temps, mélanger la farine, le sucre puis incorporer les œufs d'un seul coup, verser
ensuite le lait bouillant.
Mélanger doucement afin d'obtenir une pâte fluide comme une pâte à crêpes, laisser refroidir, puis
ajouter le rhum. Placer au réfrigérateur 24 heures.

Préchauffer le four à 270\degrees.
Verser la pâte bien refroidie dans les moules, en les remplissant jusqu'à 3mm du bord. Disposer les
cannelés sur la grille du four pendant 5 minutes, puis baisser le thermostat à 180\degrees\ et
continuer la cuisson pendant 1 heure : le cannelé doit avoir une croûte brune et un intérieur bien
moelleux.

\attention{Si les cannelés se colorent trop rapidement, baisser le four à 150\degrees.}

\end{nouvellerecette}
