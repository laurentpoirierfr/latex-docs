\begin{nouvellerecette}
\recette{Bœuf bourguignon}
\temps{30 minutes}{3h}

\ingredients[(pour 6 personnes)]{
\item 1,5 kg de bœuf à braiser (paleron, macreuse, gîte)
\item 200 g de lardons
\item 1 oignon
\item 2 carottes
\item 2 gousses d’ail
\item 500 g de champignons de Paris
\item 75 cl de vin rouge (Bourgogne)
\item 2 cuillères à soupe de farine
\item 2 cuillères à soupe d’huile
\item 40 g de beurre
\item 1 bouquet garni (thym, laurier, persil)
\item 12\,g de sel
\item 2\,g de poivre
}

Couper la viande en gros cubes. Dans une cocotte, faire revenir les lardons avec un peu de beurre, puis les réserver.

Faire dorer les morceaux de bœuf dans l’huile et le reste de beurre. Ajouter l’oignon émincé, les carottes en rondelles et l’ail haché. Saupoudrer de farine, mélanger et laisser roussir légèrement.

Verser le vin rouge, ajouter le bouquet garni, saler et poivrer. Remettre les lardons. Couvrir et laisser mijoter à feu doux pendant 2h30.

Nettoyer et émincer les champignons, les faire revenir à la poêle puis les ajouter dans la cocotte 30 minutes avant la fin de la cuisson.

Rectifier l’assaisonnement. Servir bien chaud, accompagné de pommes de terre vapeur ou de pâtes fraîches.

\attention{Utiliser un vin de qualité pour une sauce savoureuse. Ne pas hésiter à préparer le plat la veille, il n’en sera que meilleur.}

\end{nouvellerecette}
