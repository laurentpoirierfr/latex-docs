\documentclass[11pt,a4paper,sans]{moderncv}

\moderncvstyle{casual}


\moderncvcolor{orange}
\setlength{\hintscolumnwidth}{2cm} 

\usepackage[utf8]{inputenc}
\usepackage[scale=0.8]{geometry}
% \usepackage{helvet}

\usepackage[T1]{fontenc}
\usepackage{lmodern}
\usepackage[frenchb]{babel}
\usepackage{amsmath}

% https://borntocode.fr/latex-customisation-de-listes-a-puces/
\usepackage{enumitem}
\usepackage{pifont}

\name{Inmaculada}{Poirier}
\title{} 
\address{47, rue du stade}{79210 Saint Georges de Rex}{France}
\phone[mobile]{06 52 75 99 16}
% \phone[fixed]{01~01~88~33~55}
% \phone[fax]{02~11~22~33~44}
\email{inmaruizp@gmail.com}
% \homepage{www.pierredurand.com}
% \social[linkedin]{pierre.durand}
% \social[twitter]{pierre.durand}
% \social[github]{pierre.durand}
% \extrainfo{informations complémentaires.}
\photo[100pt][0.4pt]{photo.png}
% \quote{Encore un titre}


\begin{document}

\makecvtitle

%%%%%%%%%%%%%%%%%%%%%%%%%%%%%%%%%%%%%%%%%
\section{Profil professionnel}

Professionnelle dotée d’un grand sens de l’organisation, je sais m’intégrer sans difficulté dans des nouveaux
contextes de travail grace à un véritable esprit d’équipe et à d’excellentes capacités d’écoute et de
communication.

%%%%%%%%%%%%%%%%%%%%%%%%%%%%%%%%%%%%%%%%%
\section{Interruption d'activité (2012-2022)}

Pendant cette période, j'ai suivi mon conjoint en France, ce qui a nécessité une réorientation personnelle et professionnelle. 
J'ai consacré du temps à l'apprentissage de la langue française, afin de m'intégrer pleinement dans notre nouvelle vie. 
En parallèle, je me suis activement investi dans la vie locale, participant à diverses activités communautaires et bénévoles. 
Enfin, j'ai profité de cette période pour élever mon fils, lui offrant un environnement stable et propice à son développement.


%%%%%%%%%%%%%%%%%%%%%%%%%%%%%%%%%%%%%%%%%
\section{Compétences Diverses}

\begin{itemize}[label=\textbullet, font=\LARGE \color{orange}]
  \item \textbf{Polyvalence administrative} : Capacité à gérer diverses tâches administratives telles que l'accueil, la gestion des appels téléphoniques, la rédaction de documents et le suivi des dossiers.
  \item \textbf{Service client de qualité} : Aptitude à fournir un service client exceptionnel en offrant des conseils personnalisés, en répondant aux besoins des clients et en assurant un suivi efficace.
  \item \textbf{Communication efficace} : Compétence à communiquer clairement et efficacement avec les clients, les collègues et les partenaires, tant à l'oral qu'à l'écrit.
  \item \textbf{Organisation et gestion du temps} : Capacité à organiser les tâches, à établir des priorités et à gérer efficacement son temps pour respecter les délais et atteindre les objectifs fixés.
  \item \textbf{Esprit d'équipe et collaboration} : Capacité à travailler efficacement en équipe, à partager les informations, à collaborer avec les collègues et à contribuer au succès collectif.
  \item \textbf{Adaptabilité et flexibilité} : Capacité à s'adapter rapidement aux changements, à gérer les situations imprévues et à s'ajuster aux différentes exigences du poste.
  \item \textbf{Orientation vers les résultats} : Engagement à atteindre et dépasser les objectifs fixés, à fournir un service de qualité et à contribuer à la réussite de l'équipe ou de l'organisation.
  \item \textbf{Maîtrise des outils informatiques} : Utilisation efficace des outils informatiques et des logiciels spécifiques à votre domaine d'activité, tels que les logiciels de gestion clientèle et les applications bureautiques.
  \item \textbf{Analyse et résolution de problèmes} : Capacité à analyser les situations, à identifier les problèmes potentiels et à proposer des solutions appropriées pour les résoudre.
  \item \textbf{Sens de l'organisation et de la rigueur} : Aptitude à organiser efficacement son travail, à respecter les procédures et les normes de qualité, et à assurer un suivi minutieux des dossiers.
\end{itemize}


%%%%%%%%%%%%%%%%%%%%%%%%%%%%%%%%%%%%%%%%%
\section{Parcours professionel}

\cventry{01/2023 - en cours}{Assistante Secretaire}{Mairie}{Saint Georges de Rex}{}{%
% \begin{itemize}[label=\textbullet, font=\LARGE \color{orange}]
% \item \textbf{Gestion des élections} : Participation à l'organisation des scrutins électoraux, préparation des listes électorales, et assistance lors des opérations de vote.
% \item \textbf{Support logistique} : Organisation des événements municipaux (cérémonies, manifestations culturelles), gestion des salles communales et du matériel.
% \end{itemize}
}


\cventry{09/2022 11/2022}{Chargée de clientèle assurance (Stage)}{Maaf}{Niort}{}{%
% \begin{itemize}[label=\textbullet, font=\LARGE \color{orange}]%
%   % \item  Prise de rendez-vous clients, identification des besoins, élaboration des offres
%   % commerciales, établissement des contrats.
%   % \item Renseignements et conseils aux clients.
%   \item \textbf{Accueil et conseil aux clients} : Accueil des clients en agence, prise en charge de leurs demandes, et offre de conseils sur les produits d'assurance en fonction de leurs besoins spécifiques.
%   \item \textbf{Gestion des appels téléphoniques} : Réception des appels entrants, réponse aux questions des clients, et prise de rendez-vous avec les conseillers en assurance si nécessaire.
%   \item \textbf{Traitement des demandes clients} : Gestion des demandes de renseignements, des modifications de contrats, des déclarations de sinistres, et des demandes de résiliation.
%   \item \textbf{Élaboration de devis d'assurance} : Préparation de devis personnalisés en fonction des besoins et des caractéristiques des clients, en utilisant les outils informatiques appropriés.
%   \item \textbf{Suivi des contrats d'assurance} : Vérification de la conformité des contrats, suivi des échéances de paiement, et relance des clients en cas de retard de paiement.
%   \item \textbf{Contribution à la fidélisation des clients} : Suivi régulier de la satisfaction des clients, proposition de solutions adaptées à leurs besoins, et promotion des avantages offerts par l'assureur.
%   \item \textbf{Travail en équipe} : Collaboration avec les autres membres de l'équipe chargée de clientèle, partage des informations et des bonnes pratiques pour améliorer la qualité du service.
%   \item \textbf{Formation et développement professionnel} : Participation aux formations internes sur les produits d'assurance, les techniques de vente, et la réglementation en vigueur.
%   \item \textbf{Gestion administrative} : Saisie des données clients dans les systèmes informatiques, classement et archivage des documents, et préparation des rapports d'activité.
%   \item \textbf{Respect des normes et des procédures} : Application des règles et des procédures internes en matière de vente et de gestion des contrats d'assurance, dans le respect de la législation en vigueur.
% \end{itemize}
}

\cventry{02/2011 11/2011}{Directrice de la boutique}{Ermenegildo Zegna}{Marbella}{}{%
% \begin{itemize}[label=\textbullet, font=\LARGE \color{orange}]
%   \item \textbf{Gestion des ventes et du chiffre d'affaires} : Suivi des objectifs de vente, analyse des performances financières, et mise en place de stratégies pour augmenter le chiffre d'affaires.
%   \item \textbf{Supervision de l'équipe} : Recrutement, formation et management du personnel, élaboration des plannings, et évaluation des performances.
%   \item \textbf{Service clientèle} : Assurer un service clientèle de qualité, gestion des réclamations et des retours, et fidélisation des clients.
%   \item \textbf{Gestion des stocks} : Suivi des niveaux de stock, réapprovisionnement des produits, gestion des inventaires, et coordination avec les fournisseurs.
%   \item \textbf{Merchandising et mise en avant des produits} : Organisation et agencement attractif de la boutique, mise en place de vitrines thématiques, et optimisation de l'espace de vente.
%   \item \textbf{Marketing et promotions} : Développement et mise en œuvre de campagnes promotionnelles, gestion des réseaux sociaux de la boutique, et coordination des événements en magasin.
%   \item \textbf{Analyse des tendances du marché} : Suivi des tendances de consommation, analyse de la concurrence, et adaptation de l'offre produit en conséquence.
%   \item \textbf{Gestion administrative et financière} : Tenue des registres financiers, préparation des rapports de vente, gestion des budgets, et suivi des dépenses.
%   \item \textbf{Relations avec les fournisseurs} : Négociation des conditions d'achat, suivi des commandes, et gestion des livraisons.
%   \item \textbf{Maintien de l'ordre et de la sécurité} : Assurer la propreté et l'ordre de la boutique, veiller à la sécurité des biens et des personnes, et respecter les normes d'hygiène et de sécurité.
% \end{itemize}
}

\cventry{01/2010 01/2011}{Responsable de l’agencement des points de vente}{Tommy Hilfiger}{Marbella}{}{%
% \begin{itemize}[label=\textbullet, font=\LARGE \color{orange}]
%   \item \textbf{Conception et planification des agencements} : Élaboration des plans d'agencement des points de vente, en tenant compte des flux de clients, de l'optimisation de l'espace, et de l'attrait visuel.
%   \item \textbf{Supervision des travaux d'aménagement} : Coordination des équipes de travaux, des prestataires et des fournisseurs pour assurer le respect des délais et des budgets.
%   \item \textbf{Mise en place du merchandising} : Définition et mise en œuvre des stratégies de merchandising visuel, création de vitrines attractives, et agencement des produits pour maximiser les ventes.
%   \item \textbf{Analyse des performances des points de vente} : Suivi des indicateurs de performance des magasins, analyse des ventes par emplacement et recommandation d'ajustements d'agencement pour améliorer les résultats.
%   \item \textbf{Gestion des projets d'ouverture et de rénovation} : Planification et supervision des ouvertures de nouveaux points de vente et des projets de rénovation, coordination des équipes internes et externes.
%   \item \textbf{Veille concurrentielle et tendances du marché} : Suivi des tendances en matière de design de points de vente, analyse des pratiques de la concurrence, et proposition d'innovations pour les magasins.
%   \item \textbf{Formation et support aux équipes de vente} : Formation des équipes de vente sur les principes de merchandising et d'agencement, fourniture de supports visuels et de guides pratiques.
%   \item \textbf{Gestion des budgets d'agencement} : Élaboration et suivi des budgets pour les projets d'agencement, négociation avec les fournisseurs et les prestataires pour optimiser les coûts.
%   \item \textbf{Respect des normes de sécurité et d'accessibilité} : Assurer que les agencements respectent les normes de sécurité, d'hygiène et d'accessibilité, et effectuer des contrôles réguliers.
%   \item \textbf{Communication et coordination interservices} : Collaboration étroite avec les services marketing, achats, et logistique pour assurer une cohérence entre l'agencement des points de vente et les stratégies globales de l'entreprise.
% \end{itemize}
}

\cventry{11/2007 11/2009}{Directrice de la boutique}{Hermés}{Marbella}{}{%
% \begin{itemize}[label=\textbullet, font=\LARGE \color{orange}]
% \item x
% \end{itemize}
}

\cventry{10/2005 11/2007}{Directrice de la boutique}{Carolina Herrera}{Marbella}{}{%
% \begin{itemize}[label=\textbullet, font=\LARGE \color{orange}]
% \item x
% \end{itemize}
}

\cventry{02/2005 10/2005}{Vendeuse}{Corte Ingles}{Marbella}{}{%
% \begin{itemize}[label=\textbullet, font=\LARGE \color{orange}]
%   \item \textbf{Accueil et conseil personnalisé} : Accueil chaleureux des clients, évaluation de leurs besoins et préférences, et offre de conseils personnalisés pour les aider dans leurs choix.
%   \item \textbf{Présentation des produits} : Explication détaillée des caractéristiques, des avantages et des particularités des produits de luxe, mise en avant des matériaux, de la fabrication et de l'histoire de la marque.
%   \item \textbf{Fidélisation de la clientèle} : Création et entretien de relations durables avec les clients, suivi personnalisé après-vente, et envoi de communications exclusives (invitations à des événements, lancements de produits).
%   \item \textbf{Gestion des ventes} : Réalisation des transactions de vente, manipulation des systèmes de caisse, gestion des paiements, et suivi des procédures de facturation et d'encaissement.
%   \item \textbf{Merchandising visuel} : Contribution à l'agencement et à la présentation des produits en magasin, création de vitrines attractives, et maintien d'un espace de vente soigné et luxueux.
%   \item \textbf{Gestion des stocks} : Suivi des niveaux de stock, réapprovisionnement des produits, réalisation des inventaires, et coordination avec le service logistique.
%   \item \textbf{Formation continue} : Participation aux formations sur les produits, les tendances du marché, et les techniques de vente, pour rester à jour et offrir un service de haute qualité.
%   \item \textbf{Service après-vente} : Gestion des retours et des échanges, résolution des réclamations avec professionnalisme, et suivi des réparations ou des services complémentaires.
%   \item \textbf{Participation aux événements de la boutique} : Assistance lors des événements VIP, des lancements de produits, et des promotions spéciales, en offrant un service d'exception aux invités.
%   \item \textbf{Contribution à l'atteinte des objectifs de vente} : Travail en équipe pour atteindre et dépasser les objectifs de vente mensuels et annuels, et proposition d'initiatives pour améliorer les performances commerciales.
% \end{itemize}
}

\cventry{10/2000 12/2004}{Assistante dans un bureau d’architecte}{Daniel Trujillano}{Estepona}{}{%
% \begin{itemize}[label=\textbullet, font=\LARGE \color{orange}]
%   \item \textbf{Gestion administrative} : Réception et distribution des appels téléphoniques, gestion des courriels, et accueil des visiteurs. Préparation et gestion des documents administratifs, correspondances, et rapports.
%   \item \textbf{Support aux projets} : Assistance dans la préparation des dossiers de présentation, des plans, et des documents techniques. Suivi des différentes phases des projets et mise à jour des fichiers de projet.
%   \item \textbf{Organisation des réunions} : Planification des réunions, préparation des ordres du jour, prise de notes pendant les réunions, et rédaction des comptes-rendus.
%   \item \textbf{Gestion des agendas} : Coordination et gestion des agendas des architectes, planification des rendez-vous et des déplacements.
%   \item \textbf{Relations avec les clients et les partenaires} : Coordination des communications avec les clients, les entrepreneurs, les ingénieurs, et les autres parties prenantes. Organisation des réunions de suivi et des visites de chantier.
%   \item \textbf{Gestion des documents et archives} : Classement et archivage des plans, des dessins, des contrats, et des autres documents relatifs aux projets. Maintenance des bases de données et des systèmes de gestion documentaire.
%   \item \textbf{Support financier} : Assistance à la préparation des devis, des factures, et des suivis de paiement. Suivi des budgets de projets et gestion des dépenses courantes.
%   \item \textbf{Préparation des présentations} : Aide à la création de présentations visuelles et de supports de communication pour les clients et les réunions internes.
%   \item \textbf{Suivi des autorisations et des permis} : Assistance dans la préparation et le dépôt des demandes de permis de construire et des autres autorisations nécessaires. Suivi des démarches administratives auprès des autorités locales.
%   \item \textbf{Coordination logistique} : Organisation des déplacements et des hébergements pour les membres de l'équipe, préparation et gestion des déplacements sur les sites de projets.
%   \item \textbf{Soutien à la communication interne et externe} : Mise à jour du site internet de l'agence, gestion des réseaux sociaux, et participation à la création de matériel promotionnel et de communication.
% \end{itemize}
}

%%%%%%%%%%%%%%%%%%%%%%%%%%%%%%%%%%%%%%%%%
\section{Formation}

\cventry{2022}{Chargée de Clientèle en Assurances, Certification Professionnelle du Livret ORIAS II}{}{Niort}{}{}
\cventry{1991}{BTS Secrétariat}{}{Espagne}{}{}
\cventry{1988}{Bac généraliste}{}{Espagne}{}{}

\cventry{}{\begin{itemize}[label=\textbullet, font=\LARGE \color{orange}]
  \item \textbf{Pack Office}
  \item \textbf{Familiarisée avec l'usage d'internet}
\end{itemize}
}{}{}{}{}


%%%%%%%%%%%%%%%%%%%%%%%%%%%%%%%%%%%%%%%%%
\section{Langues}

\cvitemwithcomment{Espagnol}{Maternelle}{}
\cvitemwithcomment{Anglais}{Operationnel}{}

%%%%%%%%%%%%%%%%%%%%%%%%%%%%%%%%%%%%%%%%%
\section{Centres d'intérêts}
\cventry{}{}{}{}{}{%
\begin{itemize}[label=\textbullet, font=\LARGE \color{orange}]
  \item \textbf{Randonnée}
  \item \textbf{Jardinage}
  \item \textbf{Lecture}
\end{itemize}}


%%%%%%%%%%%%%%%%%%%%%%%%%%%%%%%%%%%%%%%%%
\end{document}